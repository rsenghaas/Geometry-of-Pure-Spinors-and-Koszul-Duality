\section{What's the plan?}

We goal is to make precise and generalize the statement of the derived pure
spinor formalism in \cite{EHS22}. The derived pure spinor equivalence says that
for $\fp$ a super-poincar\'e algebra of the form 
\begin{equation}
    \fp = \fg_0 \oplus \fn
\end{equation}
with $\fn = \fn_1 \oplus \fn_2$ a super-translation algebra and $\fg_0 = \so (d)
\times \fr$ such that $\fp$ is a graded Lie algebra.

The statement of the derived pure spinor correspondence is 
\begin{comment}
Accordingly, instead of considering only $R/I$-modules one considers
$\mbox{CE}^\bullet (\mathfrak{t})$-modules as input: geometrically, this amount
to consider modules on the derived scheme $\mbox{Spec} (\mbox{CE}^\bullet
(\mathfrak{t}))$ instead of modules on the nilpotence variety $Y = \mbox{Spec}
(R/I).$ In particular, the following holds.
\end{comment}

\begin{theorem}[\cite{EHS22}] \label{derivedPS} Given a super
Poincar\'e algebra $\mathfrak{p} = \mathfrak{g}_0 \oplus \mathfrak{n}$, the pure spinor functor defines an equivalence of dg categories 
\begin{equation}
\mbox{\emph{CE}}^\bullet (\mathfrak{t})\mbox{\emph{-\sffamily{Mod}}}^{\mathfrak{p}_0} \cong \mbox{\emph{\sffamily{Mult}}}^{\mathpzc{strict}\mbox{\tiny{-}}\mathpzc{ob}}_{\mathfrak{g}}
\end{equation}
between the category of $\mathfrak{g}_0$-equivariant $\mbox{\emph{CE}}^\bullet
(\mathfrak{n})$-modules and the full dg subcategory of the category of multiplets whose objects are strict multiplets.
\end{theorem}
\noindent The upshot of the theorem is twofold: on the one hand, from a physical
point of view, it says that (up to quasi-isomorphism) all multiplets can be
constructed via a suitable derived ``enhancement'' of the pure spinor formalism.
On the other hand, mathematically, it relates a geometric category, to a
representation-theoretic one, in a Koszul duality-like fashion. 
The first goal is to make the relation to Koszul duality precise and concret.

